\documentclass[11pt,a4paper]{article}
\usepackage[left=2.5cm,right=2.5cm,top=3cm,bottom=3.5cm]{geometry}
\usepackage[ngerman]{babel}
\usepackage[utf8]{inputenc}
\usepackage{graphicx}
\usepackage{amsfonts} 
\usepackage{svg}
\usepackage{amsmath}
\usepackage[rightcaption]{sidecap}
\begin{document}
 
 \begin{center}
  {\scshape\LARGE Proinformatik VII \par}
  \vspace{1cm}
  {\scshape\Large C-Kurs \par}
  \vspace{1.5cm}
  {\huge\bfseries \"Ubungsblatt 1 \par}
  \vspace{2cm}
     {\large \itshape{Clemens Schumann}\/ \par}
  \vspace{0.5cm}
  {clemensrubenschumann@googlemail.com}
  \vfill
  betreut von\par
  \textsc{Leonard K\"onig}
  \vfill
  {\Large 27.03.2018}
 
 \end{center}
 
 \thispagestyle{empty}
 
 \newpage
 \setcounter{page}{1}

    \section{Aufgabe 3}
    \subsection{Schaltjahrberechnung}
    Ein Schaljahr ist ein Jahr f\"ur das gilt:
    \begin{equation}
        L(Y) = 4 | Y \wedge 100 \nmid Y \vee 400 | Y
    \end{equation}
    Folgendes C- Programm berechnet unter Eingabe eines Datums, ob dieses ein
    Schaltjahr ist:
    


\end{document}
